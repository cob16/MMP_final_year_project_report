\chapter{Implementation}

% \todo[inline]{
% The implementation should look at any issues you encountered as you tried ,to implement your design. During the work, you might have found that elements of your design were unnecessary or overly complex; perhaps third party libraries were available that simplified some of the functions that you intended to implement. If things were easier in some areas, then how did you adapt your project to take account of your findings?

% It is more likely that things were more complex than you first thought. In particular, were there any problems or difficulties that you found during implementation that you had to address? Did such problems simply delay you or were they more significant? 

% You can conclude this section by reviewing the end of the implementation stage against the planned requirements. 

% \begin{itemize}
%     \item lazy mode
%     \item blog itterations 
% \end{itemize}
% }

An iteration time-span of two weeks was used, from the start of the project on \formatdate{30}{01}{2017} to the end of the second term \formatdate{03}{04}{2017}. This gave me five two week long iterations. I have reflected on this later in my evaluation.

\section{Iteration 0}
\subsection*{\formatdate{30}{01}{2017} -- \formatdate{06}{02}{2017}}

The zeroth iteration was the exploration phase of the project. The majority of the scratch programs and story (CRC) cards were written during this time. Assumptions from Ben Cooper's report\cite{ben_report} were verified by using an oscilloscope in a serial decoding mode. I also studied the radio itself as well as its service and user manuals~\cite{user_manual}. 

Access to the radio was delayed for the first half of this week. Instead I focused on looking at integrating my application into existing solutions. The commonly used application for users that desire computer radio control was the application Hamlib\cite{hamlib}. I first read its documentation and contacted the developer mailing list to discus the suitability of a patch for the \gls{8900}. Hamlib provides a generic API for over 200 ham radios. It also has a client/server mode for TCP Networking making remote control easier, as well as bindings for C, C++, Perl and Python.

After some discussion with the developers of Hamlib they thought that as the radio would require more than purely a USB cable to interface with the computer, it was not really something that should be in their system. They helpfully directed me to other projects that had similar purposes written in Python and Golang (hence their inclusion when deciding language choice in section~\ref{section:language_choice}). Before this decision the project was going to be a patch to Hamlib to add a new ``backend''. 

\section{Iteration 1}
\subsection*{\formatdate{06}{02}{2017} -- \formatdate{20}{02}{2017}}
Iteration 1 was my first ``real'' iteration in that it involved implementation. In this stage I started my initial Git repository commit, designing the file structure of the application. For the Makefiles I used CMake, an open-source tool that creates Makefiles. This was because the test suite I was using is an uncommon dependency and CMake has the ability to download dependencies if needed. CMake also generates the necessary Makefiles based on the current state of libraries on the system. 

In this iteration I planned to work on the first feature, setting a frequency. However there was some way to go before this was possible. First I created the bulk of packet.h, my representation of the 8 bit segments in the packet. In this I made the assumption that there are at 8 bits in a byte. This cost me portability to very old systems that do not allocate 8 bits to a byte. However I suspect if this becomes an issue I can create a alternate definition that is swapped in at compile time if the assumption is incorrect.

In addition I started creating constants needed for creating packets, such as the default values and their allowed ranges. The application release at the end of this iteration was that it supplied a valid packet to the radio. This meant that it could be run without the original radio head attached. Before this, the radio would have automatically shutdown due to inactivity and while there was no way to do this in real-time, the packet could be configured to be ``presssing'' a button on the radio.

I had not met my first feature yet but this was likely because setting the frequency was the most difficult feature and the beginnings of the application had to be created first. I continued this feature in the second iteration.

\section{Iteration 2}
\subsection*{\formatdate{20}{02}{2017} -- \formatdate{6}{03}{2017}}

For the second iteration I naturally wanted to focus on the most undeveloped part of the program, the display packet so that it too could be represented in my program just like the control packet. I however instead I stuck to my first stated goal of setting the frequency.

Before implementing the packet sending thread that was required to send a stream  of different packets continuously, I worked on adding linting to my CI to avoid potential bugs. Debugging threads is often a major pain point in programming that I wanted to avoid. This had an immediate pay off by finding a number of potential bugs in my application.

It was as this point I encountered the worst bug of the project. It stemmed from me setting, but then not resetting a compiler option (pragma\_pack) in my program. This resulted in a pointer getting mangled thereby causing a segfault in a completely unrelated library that was used in the main application. It was likely that the library relied on the completely opposite assumption of how much padding there was between memory. While unit tests did not help me find the bug they helped me rule out a large portion of my codebase as a culprit. 

For the packet sending thread, I used a linked list as a way to queue packets to be sent to the radio. In C it is often common to make intrusive data structures. In this case that would mean placing pointers to the next and previous nodes in the queue within the control packet struct itself. The non-intrusive alternative is to have a second struct that holds a pointer to the control packet and the information for the list. The benefit of the intrusive data structure is that memory allocation is only done once (with malloc as there is only one struct instead of two), at a large scale, unnecessary mallocs have a significant impact on speed~\cite{curl_mallocs}, with the trade off being that the structure is now less easy to use in another linked list implementation. 

As this was for a library it's possible that other programmers would not uses my sender queue and instead only want to represent control packets as compactly as possible. In addition the current user actions though the shell did not invoke hundreds of malocs therefore the performance benefits are negligible. Therefore at the current time I opted for the non-intrusive method. If in the future performance becomes an issue then this would be something consider but at the moment this is unnecessary pre-optimisation.

In some testing with timings I founds that the radio did not respond to keypresss that were for less than 50ms. This was likely due to software accounting bouncing of the voltages from the buttons. There is some noise created temporally by the circits when the button moved, so the program instead takes the average reading inside 50ms. The radio did this no matter how many packets were send inside this timne. I concluded that it was useless to send packets as fast as the head does and instead added longer waiting periods between packets. This lowered the CPU usage of the program by 90\%.

At the end of this iteration the application was able to create a stream of packets that would successfully dial the requested numbers into the radio. 

\section{Iteration 3}
\subsection*{\formatdate{6}{03}{2017} -- \formatdate{20}{03}{2017}}
Iteration 3 focused on users setting the volume, \gls{ptt} and \gls{squelch} controls. These features were bundled together due to there similar implementation as they only required updating a single packets. Both left and right receivers can be set independently. Input is validated to be within the expected range of the radio (and is tested accordingly).

Logging output levels were added so that output could be filtered by importance, otherwise the output of the program was too verbose. At this stage a rudimentary prompt was created a for users to interact with the program in real-time. \todo[]{add to this}

\section{Iteration 4}
\subsection*{\formatdate{20}{03}{2017} -- \formatdate{03}{04}{2017}}

This Iteration focused on features that required information displayed on the screen. However a major refactor of the library  was undertaken so that it could be properly included through a single header file. Before this the library was included in the main application through its C files, meaning that the main program compiled its own version of the library. The end result was a much simpler dependency tree.

Reading the frequency display packets first involved mapping out all the bits in the packet used for each digit. The actual mapping process had been done by an unknown author and posted to the Internet\todo[]{cite this}. This was complex as the required sequence of bits in the packet were scattered throughout. A large number of constants were defined for these mappings so that the functions remained legible. \todo[]{talk about bit mapping macro in code}


After the bits had been retrieved they had to then be decoded back into the digit that they represented on the screen. A 13x10 truth table was devised to decode the 13 segment displays that represent the 10 digits. The radio will sometimes display letters here under some configurations in order to show station names (named by the user). I deemed decoding this unnecessary as the system only needed to read the frequencies. If letters are shown on the screen -1 is returned to signify an error.

\begin{table}[h!]
\vspace{0.5cm}
\centering
\begin{tabular}{l|l|l|l|l|l|l|l|l|l|l|l|l|l}
\# & \textbf{A} & \textbf{B} & \textbf{C} & \textbf{D} & \textbf{E} & \textbf{F} & \textbf{G} & \textbf{H} & \textbf{I} & \textbf{J} & \textbf{K} & \textbf{L} & \textbf{M} \\
\hline
\textbf{0} & 1 & 1 & 1 & 1 & 1 & 1 & 0 & 0 & 1 & 0 & 1 & 1 & 1 \\
\hline
\textbf{1} & 1 & 0 & 1 & 0 & 0 & 0 & 0 & 0 & 0 & 0 & 0 & 0 & 0 \\
\hline
\textbf{2} & 0 & 1 & 1 & 0 & 1 & 1 & 0 & 0 & 1 & 0 & 0 & 1 & 0 \\
\hline
\textbf{3} & 1 & 1 & 1 & 0 & 1 & 1 & 0 & 0 & 1 & 0 & 0 & 0 & 0 \\
\hline
\textbf{4} & 1 & 1 & 1 & 0 & 0 & 1 & 0 & 0 & 0 & 0 & 0 & 0 & 1 \\
\hline
\textbf{5} & 1 & 1 & 0 & 0 & 1 & 1 & 0 & 0 & 1 & 0 & 0 & 0 & 1 \\
\hline
\textbf{6} & 1 & 1 & 0 & 0 & 1 & 1 & 0 & 0 & 1 & 0 & 0 & 1 & 1 \\
\hline
\textbf{7} & 1 & 0 & 1 & 0 & 1 & 0 & 0 & 0 & 0 & 0 & 0 & 0 & 0 \\
\hline
\textbf{8} & 1 & 1 & 1 & 0 & 1 & 1 & 0 & 0 & 1 & 0 & 0 & 1 & 1 \\
\hline
\textbf{9} & 1 & 1 & 1 & 0 & 1 & 1 & 0 & 0 & 1 & 0 & 0 & 0 & 1
% \hline
\end{tabular}
\label{13_segment_truth}
\caption[13 Segment Truth table]{A Truth table that decodes the 13 bit segment display to their corresponding numbers. This had to be made as it differs from generic segmented displays.}
\vspace{0.5cm}
\end{table}

Other features such as discerning which receiver was selected as the ``main'', reading the set power level and if each receiver was ``busy'' were implemented. These function simply checked a single bit in the display packets so were trivial to implement.

When testing this feature there were a number of behaviours of the radio to account for. Radio transmission is only permitted on a subset of the frequencies that it is capable of receiving on. To account for this frequency validation on user input was implemented. The application had to be aware if the radio was currently permitted to transmit on a given frequency. 

Frequency vvalidationhad not only to be added to the frequency functions but also the \gls{ptt} and Power commands. This was an undefined behaviour that was not in the user manual~\cite{user_manual} of the radio. The power level cannot be changed when on frequency that does not allow transmission. Furthermore the radio does not notify the user in the usual way, an error sound followed with the text ``Error'' written on the screen. Instead the radio does nothing when the button is pressed. The only workaround for this was to check if the current frequency allowed transmission before setting power levels. Users are given a warning message with an explanation if this was not the case.

The next feature was to be able to boot the radio from the software. This can only be done by connecting the 5th line on the serial connector cable (the power switch line) to the 3rd line (ground). I implemented a signal that can be sent via the \gls{dtr} pin.  When given the ``--dtr-on'' flag the application attempts to turn on the radio by signalling on this pin. If after 3 failed attempts and the radio an still not on, then the application shuts down. This was implemented is software however was partly abandoned due to the the increase in complexity of hardware required. The user would require extra circuitry to read this signal and ground the pins. It may be possible to achieve this using a single \gls{MOSFET} to complete the circuit when the \gls{dtr} is set to high.

The rest of the time was used to refactor the shell. Each command was extracted into its own function instead of one massive case statement. An array of structs was created that held a pointer to that function, the function keyword, the number of arguments and help text. The shell then iterated through the array to match with the keyword and number of required arguments. This implementation made the shell very modular and also allowed for meta functions such as the help command to be easily added and implemented.