\chapter{Implementation}

% \todo[inline]{
% The implementation should look at any issues you encountered as you tried ,to implement your design. During the work, you might have found that elements of your design were unnecessary or overly complex; perhaps third party libraries were available that simplified some of the functions that you intended to implement. If things were easier in some areas, then how did you adapt your project to take account of your findings?

% It is more likely that things were more complex than you first thought. In particular, were there any problems or difficulties that you found during implementation that you had to address? Did such problems simply delay you or were they more significant? 

% You can conclude this section by reviewing the end of the implementation stage against the planned requirements. 

% \begin{itemize}
%     \item lazy mode
%     \item blog itterations 
% \end{itemize}
% }

An iteration time-span of two weeks was used, from the start of the project on \formatdate{30}{01}{2017} to the end of the second term \formatdate{03}{04}{2017}. This gave five two week long iterations. This decision is later reflected on in the evaluation.

\section{Iteration 0}
\subsection*{\formatdate{30}{01}{2017} -- \formatdate{06}{02}{2017}}

The zeroth iteration was the exploration phase of the project. The majority of the scratch programs and story (CRC) cards were written during this time. Assumptions from Ben Cooper's report\cite{ben_report} were verified by using an oscilloscope in a serial decoding mode. Also studied was the radio itself, including its service and user manuals~\cite{user_manual}. 

As access to the radio was delayed for the first half of this week, focus was given to the possibility of integrating a solution into existing applications instead of a bespoke piece of software. The commonly used application for users that desire computer radio control was the application Hamlib\cite{hamlib}. The Hamlib documentation was studied and the developer mailing list utilised to discuss the suitability of a patch for the \gls{8900}. Hamlib provides a generic API for over 200 ham radios. It also has a client/server mode for TCP networking making remote control easier. In addition there are language bindings for C, C++, Perl and Python.

After some discussion with the developers of Hamlib it became apparent that the radio would require more than purely a USB cable to interface with the computer, and therefore it was not really something that should be in their codebase. They helpfully referenced many other projects that had been created for similar purposes on other radios. These were written in Python and Golang (hence their inclusion when deciding language choice in section~\ref{section:language_choice}). Before this decision the project was going to be a patch to Hamlib to add a new ``backend''. 

\section{Iteration 1}
\subsection*{\formatdate{06}{02}{2017} -- \formatdate{20}{02}{2017}}
The first iteration could be considered the first ``real'' iteration as it involved actual software implementation. The initial Git commit was created and the overall file structure of the application was designed. For Makefiles CMake\cite{cmake} was used, an open-source tool to generate the Makefiles. Cmake was first used because the test suite was sometimes an uncommon dependency on machines and CMake had the ability to download dependencies if needed. CMake also generates the necessary Makefiles based on the current state of libraries on the system. Later this system became invaluable to link and generate the multiple executables required.

This iteration was planned around developing the first feature, setting the radio frequency. However there was some way to go before this was possible. First, the bulk of ``packet.h'' had to be made to represent each 8 bit segment in a packet. An assumption was made that there are 8 bits in a byte. This will cost some portability to very old systems that do not allocate 8 bits to a byte. However, if this becomes an issue an alternate definition can be swapped in at compile time using a pre-processor rule that checks this the assumption is incorrect.

Additionally, the necessary constants that were expected in control packets were made, for example such as the default values and their allowed ranges. The end result of this iteration was that it could supply valid packets to the radio. This meant that it could be run without the original radio head attached. Before this, the radio would have automatically shutdown due to inactivity and while there was no way to do this in real-time, the packet could be configured to be ``presssing'' a button on the radio. While this iteration had not met its goal this was to be somewhat expected for the first feature. Firstly because setting the frequency was one of the most difficult features and secondly the beginnings of the application had to be created first. This feature was continued in the second iteration.

\section{Iteration 2}
\subsection*{\formatdate{20}{02}{2017} -- \formatdate{6}{03}{2017}}

For this iteration was there was temptation to start work on the most undeveloped part of the program, the display packet. So that it too could be represented like the control packet program. However the project stuck to its original stated goal of developing to set the frequency.

A packet sending thread was required to send the required stream of packets continuously. However work was first done on adding \gls{linting} to the Jenkins CI to avoid potential bugs. As debugging threads is often a major pain point in programming, therefore the project should attempt to avoid this by testing often. This had an immediate pay off by finding a number of potential bugs within the application.

It was as this point the worst bug of the project was encountered. The effect was a pointer getting mangled thereby causing a segfault in a completely unrelated library that was used in the main application. This stemmed from setting, but then not resetting a compiler option (pragma\_pack) in the program. This changes the amount of padding in memory between blocks of used memory. It was likely that the library relied on the completely opposite assumption of how much padding there was between memory. While unit tests did not help find the bug they helped to rule out a large portion of the codebase as a culprit. 

For the packet sending thread, a linked list was used as a way to queue packets to be sent to the radio. In C it is often common to make intrusive data structures. A intrusive data structure is when extra fields inside the struct are used to help with the the queue implementation. In this case that would mean placing pointers to the next and previous nodes in the queue within the control packet struct itself. The non-intrusive alternative is to have a second struct that holds a pointer to the control packet and along with the information to the next item in the list. The benefits of an intrusive data structure is that memory allocation is only done once (with Malloc as there is only one struct instead of two). At a large scale, unnecessary use of Malloc could have a significant impact on speed~\cite{curl_mallocs}, with the trade off being that the structure is now less easy to use in another linked list implementation. 

As this queue was to be used in a library it is possible that other programmers would not want to use the sender queue. Instead they may wish to want to represent control packets as compactly as possible. In addition, the current actions through the user shell did not invoke more than a few Malloc calls at a time. Therefore, the performance benefits were negligible, so the non-intrusive method was chosen. If in the future, if performance becomes an issue then this would be something to consider. However, currently this would be unnecessary pre-optimisation.

After some manual testing of the application, the radio was found to not respond to keypresses that were for less than 50ms. After some investigation this was likely due to software accounting bouncing of the voltages from the buttons. There is some noise created temporally by the circuits when the button moved, so the program instead takes the average reading inside 50ms. The radio did this no matter how many packets were send inside this time. Therefore it was concluded that it was useless to send packets as fast as the head does and instead added a longer waiting period between packets. This lowered the CPU usage of the program by 90\%. At the end of this iteration the application was able to create a stream of packets that upon user request would successfully dial the requested numbers into the radio. 

\section{Iteration 3}
\subsection*{\formatdate{6}{03}{2017} -- \formatdate{20}{03}{2017}}
Iteration 3 focused on users setting the volume, \gls{ptt} and \gls{squelch} controls. These features were bundled together due to there similar implementation as they only required updating a single packet. Both left and right receivers can be set independently. Input is then validated to be within the expected range of the radio (and is tested accordingly). If a user gives a value out of range the closest valid number is used instead.

Logging output levels were added so that output could be filtered by importance, otherwise the output of the program was too verbose, making reading output impossible for regular users of the system. A wrapper function for ``vprintf()'' was made that took an extra Enum of logging level constants. The desired logging level can then be set with the -v flag. The use of Syslog was considered, but it was concluded that this would not be very useful for most users of the system. This is because they would have to configure the logging levels themselves in a separate system configuration file.

At this stage a rudimentary prompt was created a for users to interact with the program in real-time. Initially this was done very poorly, as a large case statement of string comparisons. This was done as this was the fastest way to implement the current needed features, and a project demonstration was at the end of the iteration. This reduced the readability and robustness of the code so was revisited latter in iteration 4. This code debt was arguably a good investment as users were able to see the intention of the final product and provide meaningful feedback early. 

\section{Iteration 4}
\subsection*{\formatdate{20}{03}{2017} -- \formatdate{03}{04}{2017}}

This Iteration focused on features that required information displayed on the screen. However a major refactor of the library was undertaken so that it could be properly included through a single header file. Before this, the library was included in the main application through its C files, meaning that the main program compiled its own version of the library. The end result was a much simpler dependency tree.

Reading the frequency display packets first involved mapping out all the bits in the packet used for each digit. The actual mapping process had been done by an unknown author and posted to the Internet\cite{8800r_reverse}. This was complex as the required sequences of bits that were scattered throughout the packet. A large number of constants were defined for these mappings so that the functions that used them remained readable. These constants were the position of the bit in the overall packet. To make reading of this code essayer, a pre-processor macro (Shown in figure~\ref{fig:packet_mapping}) was devised so that the developer could list the position using the bit and byte number.

\begin{figure}
\centering
\begin{minted}{C}
#define BIT_LOCATED_AT(byte, bit) (((byte) * 8) + (bit))

enum display_packet_bitmasks {
        //Misc
        AUTO_POWER_OFF = BIT_LOCATED_AT(0, 1),
        KEYPAD_LOCK    = BIT_LOCATED_AT(30, 4),
        SET            = BIT_LOCATED_AT(4, 0)
}
\end{minted}
\caption[Packet mapping]{An example of 3 of the 190 mappings used in the display packet.}
\label{fig:packet_mapping}
\end{figure}

After the bits had been retrieved they had to then be decoded back into the digit that they represented on the screen. A 13x10 truth table was devised to decode the 13 segment displays that represent the 10 digits. The radio will sometimes display letters here under some configurations in order to show station names (named by the user). Decoding this was deemed unnecessary, as the system only needed to read the frequencies. If letters are shown on the screen -1 is returned to signify an error.

\begin{table}[h!]
\vspace{0.5cm}
\centering
\begin{tabular}{l|l|l|l|l|l|l|l|l|l|l|l|l|l}
\# & \textbf{A} & \textbf{B} & \textbf{C} & \textbf{D} & \textbf{E} & \textbf{F} & \textbf{G} & \textbf{H} & \textbf{I} & \textbf{J} & \textbf{K} & \textbf{L} & \textbf{M} \\
\hline
\textbf{0} & 1 & 1 & 1 & 1 & 1 & 1 & 0 & 0 & 1 & 0 & 1 & 1 & 1 \\
\hline
\textbf{1} & 1 & 0 & 1 & 0 & 0 & 0 & 0 & 0 & 0 & 0 & 0 & 0 & 0 \\
\hline
\textbf{2} & 0 & 1 & 1 & 0 & 1 & 1 & 0 & 0 & 1 & 0 & 0 & 1 & 0 \\
\hline
\textbf{3} & 1 & 1 & 1 & 0 & 1 & 1 & 0 & 0 & 1 & 0 & 0 & 0 & 0 \\
\hline
\textbf{4} & 1 & 1 & 1 & 0 & 0 & 1 & 0 & 0 & 0 & 0 & 0 & 0 & 1 \\
\hline
\textbf{5} & 1 & 1 & 0 & 0 & 1 & 1 & 0 & 0 & 1 & 0 & 0 & 0 & 1 \\
\hline
\textbf{6} & 1 & 1 & 0 & 0 & 1 & 1 & 0 & 0 & 1 & 0 & 0 & 1 & 1 \\
\hline
\textbf{7} & 1 & 0 & 1 & 0 & 1 & 0 & 0 & 0 & 0 & 0 & 0 & 0 & 0 \\
\hline
\textbf{8} & 1 & 1 & 1 & 0 & 1 & 1 & 0 & 0 & 1 & 0 & 0 & 1 & 1 \\
\hline
\textbf{9} & 1 & 1 & 1 & 0 & 1 & 1 & 0 & 0 & 1 & 0 & 0 & 0 & 1
% \hline
\end{tabular}
\label{13_segment_truth}
\caption[13 Segment Truth table]{A Truth table that decodes the 13 bit segment display to their corresponding numbers. This had to be made as it differs from generic segmented displays.}
\vspace{0.5cm}
\end{table}

Other features such as discerning which receiver was selected as the ``main'', reading the set power level and if each receiver was ``busy'' were implemented. These functions simply checked a single bit in the display packets so were trivial to implement.

When testing this feature there were a number of behaviours of the radio to account for. Radio transmission is only permitted on a subset of the frequencies that it is capable of receiving on. To account for this frequency validation on user input was implemented. The application had to be aware if the radio was currently permitted to transmit on a given frequency. 

Frequency validation had not only to be added to the frequency functions but also the \gls{ptt} and Power commands. This was an undefined behaviour that was not in the user manual~\cite{user_manual} of the radio. The power level cannot be changed when on frequency that does not allow transmission. Furthermore the radio does not notify the user in the usual way, an error sound followed with the text ``Error'' written on the screen. Instead the radio does nothing when the button is pressed. The only workaround for this was to check if the current frequency allowed transmission before setting power levels. Users are given a warning message with an explanation if this was not the case.

The next feature was to be able to boot the radio from the software. This can only be done by connecting the 5th line on the serial connector cable (the power switch line) to the 3rd line (ground). A signal that could be sent via the \gls{dtr} pin was implemented so that when given the ``--dtr-on'' flag, the application attempts to turn on the radio. If after 3 failed attempts the radio has not turned on, then the application shuts down. This was implemented in software, however was partly abandoned due to the the increase in complexity of hardware required. The user would require extra circuitry to read this signal and ground the pins. It may be possible to achieve this using a single \gls{MOSFET} to complete the circuit when the \gls{dtr} is set to high.

The rest of the time was used to refactor the shell. Each command was extracted into its own function instead of one massive case statement. An array of structs was created that held a pointer to that function, the function keyword, the number of arguments and help text. The shell then iterated through the array to match with the keyword and number of required arguments. This implementation made the shell very modular and also allowed for meta functions such as the help command to be easily added and implemented.