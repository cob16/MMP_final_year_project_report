\thispagestyle{empty}

\begin{center}
    {\LARGE\bf Abstract}
\end{center}

In amateur radio, C.A.T (Computer Aided Transceiving) is the name for a radio's ability to interact with a computer, often via serial communication. This is often done so that the radio can behave like a computer peripheral. There are many applications that rely on this premise to provide a user interface on a computer screen to fully control a radio.

The \gls{8900} is a popular budget amateur radio. However, the the \gls{8900} it is missing a significant feature, C.A.T (Computer Aided Transceiving). The design of the The \gls{8900} allows users the potential to add this feature to the radio. This is due to its ability to split off the front control surface from the main body. The detachment is made via a connecting serial line and is intended to be used to mount the controls of the radio in the console of a car. 

This report details the process of research, design and development of an application that is capable of controlling the radio, by tapping into this serial connection. The application was developed as an alternative to existing commercial solutions, with further consideration given to cost constraints. The typical conditions expected of CAT capable radios gave rise to strong efficiency and portability requirements. 

The end result was an application written in C that met it's objectives to produce a \gls{mvp}. The produced application has been evaluated for its quality and suitability as a solution.